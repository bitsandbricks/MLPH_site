\documentclass[]{book}
\usepackage{lmodern}
\usepackage{amssymb,amsmath}
\usepackage{ifxetex,ifluatex}
\usepackage{fixltx2e} % provides \textsubscript
\ifnum 0\ifxetex 1\fi\ifluatex 1\fi=0 % if pdftex
  \usepackage[T1]{fontenc}
  \usepackage[utf8]{inputenc}
\else % if luatex or xelatex
  \ifxetex
    \usepackage{mathspec}
  \else
    \usepackage{fontspec}
  \fi
  \defaultfontfeatures{Ligatures=TeX,Scale=MatchLowercase}
\fi
% use upquote if available, for straight quotes in verbatim environments
\IfFileExists{upquote.sty}{\usepackage{upquote}}{}
% use microtype if available
\IfFileExists{microtype.sty}{%
\usepackage{microtype}
\UseMicrotypeSet[protrusion]{basicmath} % disable protrusion for tt fonts
}{}
\usepackage{hyperref}
\hypersetup{unicode=true,
            pdftitle={Machine Learning para Humanxs},
            pdfauthor={Victoria O'Donnell y Antonio Vazquez Brust},
            pdfborder={0 0 0},
            breaklinks=true}
\urlstyle{same}  % don't use monospace font for urls
\usepackage{natbib}
\bibliographystyle{apalike}
\usepackage{longtable,booktabs}
\usepackage{graphicx,grffile}
\makeatletter
\def\maxwidth{\ifdim\Gin@nat@width>\linewidth\linewidth\else\Gin@nat@width\fi}
\def\maxheight{\ifdim\Gin@nat@height>\textheight\textheight\else\Gin@nat@height\fi}
\makeatother
% Scale images if necessary, so that they will not overflow the page
% margins by default, and it is still possible to overwrite the defaults
% using explicit options in \includegraphics[width, height, ...]{}
\setkeys{Gin}{width=\maxwidth,height=\maxheight,keepaspectratio}
\IfFileExists{parskip.sty}{%
\usepackage{parskip}
}{% else
\setlength{\parindent}{0pt}
\setlength{\parskip}{6pt plus 2pt minus 1pt}
}
\setlength{\emergencystretch}{3em}  % prevent overfull lines
\providecommand{\tightlist}{%
  \setlength{\itemsep}{0pt}\setlength{\parskip}{0pt}}
\setcounter{secnumdepth}{5}
% Redefines (sub)paragraphs to behave more like sections
\ifx\paragraph\undefined\else
\let\oldparagraph\paragraph
\renewcommand{\paragraph}[1]{\oldparagraph{#1}\mbox{}}
\fi
\ifx\subparagraph\undefined\else
\let\oldsubparagraph\subparagraph
\renewcommand{\subparagraph}[1]{\oldsubparagraph{#1}\mbox{}}
\fi

%%% Use protect on footnotes to avoid problems with footnotes in titles
\let\rmarkdownfootnote\footnote%
\def\footnote{\protect\rmarkdownfootnote}

%%% Change title format to be more compact
\usepackage{titling}

% Create subtitle command for use in maketitle
\providecommand{\subtitle}[1]{
  \posttitle{
    \begin{center}\large#1\end{center}
    }
}

\setlength{\droptitle}{-2em}

  \title{Machine Learning para Humanxs}
    \pretitle{\vspace{\droptitle}\centering\huge}
  \posttitle{\par}
    \author{Victoria O'Donnell y Antonio Vazquez Brust}
    \preauthor{\centering\large\emph}
  \postauthor{\par}
      \predate{\centering\large\emph}
  \postdate{\par}
    \date{2019-10-30}

\usepackage{booktabs}

\begin{document}
\maketitle

{
\setcounter{tocdepth}{1}
\tableofcontents
}
\hypertarget{para-quien-es-esto}{%
\chapter*{¿Para quién es esto?}\label{para-quien-es-esto}}
\addcontentsline{toc}{chapter}{¿Para quién es esto?}

Este libro fue escrito con una audiencia en mente formada por ??????

Esperamos que el tono introductorio del texto, así como el esfuerzo puesto en explicar los conceptos con la mayor simplicidad posible, resulten de interés para un público amplio. (decía en CDpGS, lo dejo como antecedente)

\hypertarget{antes-de-empezar}{%
\section*{Antes de empezar}\label{antes-de-empezar}}
\addcontentsline{toc}{section}{Antes de empezar}

Se requiere conocimiento básico del lenguaje de programación \texttt{R}, y del ``paquete'' de funciones para manipulación y visualización de datos llamado como \texttt{Tidyverse}. Todo ello puede adquirirse pasando un tiempo con \href{https://bit.ly/datasoc}{Ciencia de Datos para Gente Sociable}, que ademas de gratuito y disponible en línea, es el manual que sirve como base para éste que están leyendo ahora.

Para practicar los ejemplos que se explicarán a lo largo del libro es necesario instalar el \href{https://cloud.r-project.org/}{lenguaje de programación R}, y la interfaz gráfica \href{https://www.rstudio.com/products/rstudio/download/}{RStudio Desktop}.

\hypertarget{intro}{%
\chapter{Introducción}\label{intro}}

Una bonita introducción

\hypertarget{sobre-el-machine-learning}{%
\section{Sobre el Machine learning}\label{sobre-el-machine-learning}}

\hypertarget{que-es}{%
\subsection{¿Qué es?}\label{que-es}}

\hypertarget{para-que-se-usa}{%
\subsection{¿Para qué se usa?}\label{para-que-se-usa}}

\hypertarget{esto-es-inteligencia-artificial}{%
\section{\texorpdfstring{¿Ésto es \emph{Inteligencia Artificial}?}{¿Ésto es Inteligencia Artificial?}}\label{esto-es-inteligencia-artificial}}

\hypertarget{machine-learning-en-accion}{%
\chapter{Machine Learning en acción}\label{machine-learning-en-accion}}

\hypertarget{un-ejemplo-de-cabo-a-rabo-con-buen-algoritmo-para-empezar.}{%
\section{Un ejemplo de cabo a rabo con buen algoritmo para empezar.}\label{un-ejemplo-de-cabo-a-rabo-con-buen-algoritmo-para-empezar.}}

Random Trees ó Random Forest ?

\hypertarget{aprendizaje-supervisado}{%
\chapter{Aprendizaje supervisado:}\label{aprendizaje-supervisado}}

\hypertarget{guatisit}{%
\section{Guatisit}\label{guatisit}}

\hypertarget{el-yeite-del-test-el-training-y-validation.}{%
\section{El yeite del test, el training y validation.}\label{el-yeite-del-test-el-training-y-validation.}}

\hypertarget{n-fold.}{%
\subsection{N Fold.}\label{n-fold.}}

\hypertarget{aplicando-algos}{%
\section{Aplicando algos}\label{aplicando-algos}}

\hypertarget{linear-regression-learning}{%
\subsection{Linear regression learning}\label{linear-regression-learning}}

Este nos da pie para luego mostrar las ventajas de algos clasicos de ML.

\hypertarget{las-metricas-que-debes-saber-fscore-roc-pres-recall}{%
\subsection{Las métricas que debés saber: Fscore, ROC, Pres, Recall}\label{las-metricas-que-debes-saber-fscore-roc-pres-recall}}

\hypertarget{mas-aprendizaje-supervisado}{%
\chapter{Más Aprendizaje supervisado}\label{mas-aprendizaje-supervisado}}

\hypertarget{arboles-de-decision-y-la-forma-de-combinarlos.}{%
\section{Arboles de decision y la forma de combinarlos.}\label{arboles-de-decision-y-la-forma-de-combinarlos.}}

\hypertarget{ensembles}{%
\section{Ensembles}\label{ensembles}}

\hypertarget{random-forest}{%
\subsection{Random Forest}\label{random-forest}}

\hypertarget{gbm}{%
\subsection{GBM}\label{gbm}}

\hypertarget{adaboost}{%
\subsection{Adaboost}\label{adaboost}}

\# Redes neuronales

\#\# De la reg logistica a la NN

\#\# La magia no estructurada

\#\#\# audio

\#\#\# imagen

\#\#\# video

\#\#\# texto.

\hypertarget{procesamiento-de-lenguaje-natural}{%
\chapter{Procesamiento de Lenguaje Natural}\label{procesamiento-de-lenguaje-natural}}

\hypertarget{aprendizaje-no-supervisado}{%
\chapter{Aprendizaje no supervisado}\label{aprendizaje-no-supervisado}}

\bibliography{book.bib,packages.bib}


\end{document}
